\documentclass[a4paper]{article}
\usepackage[font=small,labelfont=bf]{caption}
\usepackage[T1]{fontenc}
\usepackage[utf8]{inputenc}
\usepackage[italian]{babel}
\usepackage{url} % per scrivere gli indirizzi url
\usepackage{booktabs}
\usepackage{graphicx}
\usepackage[left=1.5cm,bottom=1.5cm,right=1cm,top=1cm]{geometry}
\usepackage{frontespizio}
\usepackage{wrapfig}
\usepackage{xcolor}
\usepackage{color}
\usepackage{siunitx}  
\usepackage{hyperref}
\usepackage{listings}
\usepackage{framed} %linee nel lsting
\usepackage{parskip} %risolve i dannati indent.
\usepackage{fancyvrb}
\usepackage{graphicx}





\definecolor{comment}{RGB}{237, 141, 33}
\definecolor{backy}{RGB}{ 255, 253, 183 }
\definecolor{back}{RGB}{ 30,30,30 }
\definecolor{shadecolor}{RGB}{ 30,30,30 }
\lstset{
%rulecolor=\color{white},
numbers=left,
numberstyle=\color{black},
%backgroundcolor=\color{back},   
frame=single,
showspaces=false,
basicstyle=\ttfamily,%\color{white},
showstringspaces=false,
keywordstyle=\color{purple},       % keyword style
commentstyle=\color{comment},    % comment style
stringstyle=\color{blue}    % string literal style
}



\begin{document}

\begin{center}
\begin{Huge}
Manuale linguaggio di programmazione G8
\end{Huge}
\end{center}

\includegraphics[scale=1]{UniBG.png}
\\
\\
\begin{Large}
\begin{flushright}
Stefano Villa, Matricola 1055820

Matteo Zambelli, Matricola 1055560
\end{flushright}
\end{Large}

\newpage
\begin{LARGE}
\textbf{Indice}
\end{LARGE}

\newpage

\section{Introduzione}

\newpage

\section{Strumenti di sviluppo}

\newpage

\section{Struttura}

\newpage

\section{Grammatica G8}
La grammatica creata ha, come obiettivo, quello di riprodurre il Canvas HTML5, uno strumento utilizzato per rappresentare graficamente figure geometriche più o meno complesse all’interno di un’area di disegno appositamente realizzata.
\\
\\
Il parser presenta 3 regole principali, relative alla creazione dell’area di disegno, alla lista delle figure geometriche da realizzare e alla terminazione dell’operazione di disegno.
\\
\\
Nella classe \textit{begin} l'utente deve inserire il titolo del proprio lavoro, la larghezza dell'area di disegno e l'altezza della stessa. La sintassi è la seguente:
\begin{verbatim}
TITLE titolo DRAWSPACE WIDTH larghezza area disegno DRAWSPACE HEIGTH altezza area disegno
\end{verbatim}
Nella classe \textit{list} è presente una lista di 6 figure; l'utente deve sceglierne una tra: linea, triangolo, rettangolo, curva, cerchio ed ellisse. Ogni figura presenta dei valori obbligatori, relativi alle misure che andranno da (0, 0), angolo in alto a sinistra dello schermo, al valore massimo scelto per l’area di disegno, e dei valori facoltativi, relativi a colore, spessore, riempimento, rotazione, ecc. E’ anche possibile nominare la figura a piacimento: in caso non venisse dato un nome, questo verrà settato al valore di default “no name”.
\begin{itemize}
\item \textbf{line}: la grammatica prevede l'inserimento obbligatorio delle coordinate di partenza e di arrivo. L’utente può anche inserire, facoltativamente, il colore e lo spessore della linea: in caso non venissero inseriti manualmente, verranno utilizzati i valori di default (colore nero, spessore 1). La sintassi è la seguente:
\begin{verbatim}
LINE: (NAME nome) XSTART ascissa di partenza YSTART ordinata di partenza XEND ascissa di arrivo
YEND ordinata di arrivo (COLOR colore WIDTH spessore)
\end{verbatim}
\item \textbf{triangle}: la grammatica prevede l'inserimento obbligatorio delle coordinate dei tre vertici. L’utente può anche inserire, facoltativamente, il colore del bordo, lo spessore del bordo e il riempimento della figura: in caso non venissero inseriti manualmente, verranno utilizzati i valori di default (colore bordo nero, spessore 1, colore riempimento trasparente). La sintassi è la seguente:
\begin{verbatim}
TRIANGLE: (NAME nome) XA ascissa primo vertice YA ordinata primo vertice XB ascissa secondo
vertice YB ordinata secondo vertice XC ascissa terzo vertice YC ordinata terzo vertice (COLOR
colore bordo WIDTH spessore bordo COLORBODY colore riempimento)
\end{verbatim}
\item \textbf{rectangle}: la grammatica prevede l'inserimento obbligatorio delle coordinate dei due vertici opposti. L’utente può anche inserire, facoltativamente, il colore del bordo, lo spessore del bordo e il riempimento della figura: in caso non venissero inseriti manualmente, verranno utilizzati i valori di default (colore bordo nero, spessore 1, colore riempimento trasparente). La sintassi è la seguente:
\begin{verbatim}
RECT: (NAME nome) XSTART ascissa primo vertice YSTART ordinata primo vertice XEND ascissa terzo
vertice YEND ordinata terzo vertice (COLOR colore bordo WIDTH spessore bordo COLORBODY colore
riempimento)
\end{verbatim}
\item \textbf{curve}: la grammatica prevede l'inserimento obbligatorio delle coordinate del punto di partenza, del punto di arrivo e del punto di curvatura. L’utente può anche inserire, facoltativamente, il colore del bordo, lo spessore del bordo e il riempimento della figura: in caso non venissero inseriti manualmente, verranno utilizzati i valori di default (colore bordo nero, spessore 1, colore riempimento trasparente). La sintassi è la seguente:
\begin{verbatim}
CURV: (NAME nome) XSTART ascissa di partenza YSTART ordinata di partenza XMIDDLE ascissa punto
curvatura YMIDDLE ordinata punto curvatura XEND ascissa di arrivo YEND ordinata di arrivo
(COLOR colore bordo WIDTH spessore bordo COLORBODY colore riempimento)
\end{verbatim}
\item \textbf{circle}: la grammatica prevede l'inserimento obbligatorio delle coordinate del centro e del raggio. L’utente può anche inserire, facoltativamente, il colore del bordo, lo spessore del bordo e il riempimento della figura: in caso non venissero inseriti manualmente, verranno utilizzati i valori di default (colore bordo nero, spessore 1, colore riempimento trasparente). Altro elemento facoltativo da inserire è relativo all’angolo di partenza e di arrivo: è infatti possibile sviluppare circonferenze parziali o semicirconferenze, inserendo dei limiti nell’angolo da rappresentare; in caso non venissero inseriti manualmente, verranno utilizzati i valori di default (0 come start angle e 360 come end angle) sviluppando l’intera circonferenza. La sintassi è la seguente:
\begin{verbatim}
CIRC: (NAME nome) XCENTER ascissa dell’origine YCENTER ordinata dell’origine RADIUS raggio
(STARTANGLE angolo di partenza ENDANGLE angolo di arrivo COLOR colore bordo WIDTH spessore
bordo COLORBODY colore riempimento)
\end{verbatim}
\item \textbf{ellipse}: la grammatica prevede l'inserimento obbligatorio delle coordinate del centro, del semiasse maggiore e del semiasse minore. L’utente può anche inserire, facoltativamente, il colore del bordo, lo spessore del bordo e il riempimento della figura: in caso non venissero inseriti manualmente, verranno utilizzati i valori di default (colore bordo nero, spessore 1, colore riempimento trasparente). Altro elemento facoltativo da inserire è relativo all’angolo di partenza e di arrivo: è infatti possibile sviluppare circonferenze parziali o semicirconferenze, inserendo dei limiti nell’angolo da rappresentare; in caso non venissero inseriti manualmente, verranno utilizzati i valori di default (0 come start angle e 360 come end angle) sviluppando l’intera circonferenza. E’ inoltre possibile assegnare un valore di rotazione dell’ellisse intorno alla propria origine, ruotando di un valore positivo o negativo di tot gradi; in caso non venisse inserito manualmente verrà utilizzato il valore di default che non prevede alcuna rotazione (0 come angolo di rotazione). La sintassi è la seguente:
\begin{verbatim}
ELLIPS: (NAME nome) XCENTER ascissa dell’origine YCENTER ordinata dell’origine SEMIN asse
minore SEMAX asse maggiore (STARTANGLE angolo di partenza ENDANGLE angolo di arrivo ROTATION
angolo di rotazione COLOR colore bordo WIDTH spessore bordo COLORBODY colore riempimento)
\end{verbatim}
\end{itemize}
Nella classe \textit{end} l’utente si deve limitare a scrivere END per terminare l’esecuzione del lavoro. La sintassi è la seguente:
\begin{verbatim}
END
\end{verbatim}

\newpage

\section{Traduzione}

\subsection{Vincoli semantici}

\newpage

\section{Esempio pratico}
Come esempio pratico abbiamo realizzato una rappresentazione del viso di Topolino, personaggio realizzato nel 1928 da Walter Elias Disney e oggi simbolo della multinazionale The Walt Disney Company.
\\
\\
Di seguito mostreremo il codice necessario per realizzare il logo in Canvas HTML5, successivamente il codice nel nostro linguaggio G8 e infine il risultato grafico.

\subsection{Versione Canvas HTML5}
\begin{verbatim}
<!doctype html>
<html>
<head>
<title> Topolino </title>
<style> canvas {border: 1px #000 dotted;} </style>
<script>
window.onload = function () {

	var canvas = document.getElementById('Topolino');
	var context = canvas.getContext('2d'); 

	//CirconferenzaViso
	context.beginPath();
	var centerX = 850.0;
	var centerY = 550.0;
	var radius = 300.0;
	var startAngle = 0.0* Math.PI/180;
	var endAngle = 360.0* Math.PI/180;
	context.arc (centerX, centerY, radius, startAngle, endAngle);
	context.lineWidth = 0.0;
	context.strokeStyle= '#000000';
	context.stroke();
	context.fillStyle= '#000000';
	context.fill();
	context.closePath();

	//EllisseOrecchioSinistro
	context.beginPath();
	var centerX = 600.0;
	var centerY = 210.0;
	var radiusMax = 170.0;
	var radiusMin= 130.0;
	var rotation= -45.0*Math.PI/180;
	var startAngle=0.0*Math.PI/180;
	var endAngle=360.0*Math.PI/180;
	context.ellipse(centerX, centerY, radiusMax, radiusMin, rotation, startAngle, endAngle);
	context.lineWidth = 0.0;
	context.strokeStyle= '#000000';
	context.stroke();
	context.fillStyle= '#000000';
	context.fill();
	context.closePath();

	//EllisseOrecchioDestro
	context.beginPath();
	var centerX = 1100.0;
	var centerY = 210.0;
	var radiusMax = 170.0;
	var radiusMin= 130.0;
	var rotation= 45.0*Math.PI/180;
	var startAngle=0.0*Math.PI/180;
	var endAngle=360.0*Math.PI/180;
	context.ellipse(centerX, centerY, radiusMax, radiusMin, rotation, startAngle, endAngle);
	context.lineWidth = 0.0;
	context.strokeStyle= '#000000';
	context.stroke();
	context.fillStyle= '#000000';
	context.fill();
	context.closePath();

	//EllisseParteAltaSinistraViso
	context.beginPath();
	var centerX = 750.0;
	var centerY = 450.0;
	var radiusMax = 120.0;
	var radiusMin= 170.0;
	var rotation= 0.0*Math.PI/180;
	var startAngle=0.0*Math.PI/180;
	var endAngle=360.0*Math.PI/180;
	context.ellipse(centerX, centerY, radiusMax, radiusMin, rotation, startAngle, endAngle);
	context.lineWidth = 0.0;
	context.strokeStyle= '#FFFFFF';
	context.stroke();
	context.fillStyle= '#FFFFFF';
	context.fill();
	context.closePath();

	//EllisseParteAltaDestraViso
	context.beginPath();
	var centerX = 950.0;
	var centerY = 450.0;
	var radiusMax = 120.0;
	var radiusMin= 170.0;
	var rotation= 0.0*Math.PI/180;
	var startAngle=0.0*Math.PI/180;
	var endAngle=360.0*Math.PI/180;
	context.ellipse(centerX, centerY, radiusMax, radiusMin, rotation, startAngle, endAngle);
	context.lineWidth = 0.0;
	context.strokeStyle= '#FFFFFF';
	context.stroke();
	context.fillStyle= '#FFFFFF';
	context.fill();
	context.closePath();

	//CerchioInferioreCoperturaNero
	context.beginPath();
	var centerX = 850.0;
	var centerY = 550.0;
	var radius = 299.0;
	var startAngle = 16.2* Math.PI/180;
	var endAngle = 163.8* Math.PI/180;
	context.arc (centerX, centerY, radius, startAngle, endAngle);
	context.lineWidth = 0.0;
	context.strokeStyle= '#FFFFFF';
	context.stroke();
	context.fillStyle= '#FFFFFF';
	context.fill();
	context.closePath();

	//CurvaGuanciaSinistra
	context.beginPath();
	context.lineWidth = 0.0;
	context.strokeStyle = '#FFFFFF';
	context.moveTo( 566.0, 642.0);
	context.quadraticCurveTo( 580.0, 560.0, 672.0, 580.0);
	context.stroke();
	context.fillStyle= '#FFFFFF';
	context.fill();
	context.closePath();

	//CurvaGuanciaDestra
	context.beginPath();
	context.lineWidth = 0.0;
	context.strokeStyle = '#FFFFFF';
	context.moveTo( 1028.0, 580.0);
	context.quadraticCurveTo( 1103.0, 560.0, 1136.0, 642.0);
	context.stroke();
	context.fillStyle= '#FFFFFF';
	context.fill();
	context.closePath();

	//TriangoloCoperturaNeroCentrale
	context.beginPath();
	context.lineWidth = 0.0;
	context.strokeStyle = '#FFFFFF';
	context.moveTo(566.0, 641.0);
	context.lineTo(1134.0, 641.0);
	context.lineTo(850.0, 476.0);
	context.lineTo(566.0, 641.0);
	context.stroke();
	context.fillStyle= '#FFFFFF';
	context.fill();
	context.closePath();

	//EllisseOcchioSinistro
	context.beginPath();
	var centerX = 788.0;
	var centerY = 450.0;
	var radiusMax = 40.0;
	var radiusMin= 100.0;
	var rotation= 0.0*Math.PI/180;
	var startAngle=0.0*Math.PI/180;
	var endAngle=360.0*Math.PI/180;
	context.ellipse(centerX, centerY, radiusMax, radiusMin, rotation, startAngle, endAngle);
	context.lineWidth = 3.0;
	context.strokeStyle= '#000000';
	context.stroke();
	context.closePath();

	//EllisseOcchioDestro
	context.beginPath();
	var centerX = 912.0;
	var centerY = 450.0;
	var radiusMax = 40.0;
	var radiusMin= 100.0;
	var rotation= 0.0*Math.PI/180;
	var startAngle=0.0*Math.PI/180;
	var endAngle=360.0*Math.PI/180;
	context.ellipse(centerX, centerY, radiusMax, radiusMin, rotation, startAngle, endAngle);
	context.lineWidth = 3.0;
	context.strokeStyle= '#000000';
	context.stroke();
	context.closePath();

	//EllissePupillaSinistra
	context.beginPath();
	var centerX = 810.0;
	var centerY = 480.0;
	var radiusMax = 15.0;
	var radiusMin= 40.0;
	var rotation= 0.0*Math.PI/180;
	var startAngle=0.0*Math.PI/180;
	var endAngle=360.0*Math.PI/180;
	context.ellipse(centerX, centerY, radiusMax, radiusMin, rotation, startAngle, endAngle);
	context.lineWidth = 0.0;
	context.strokeStyle= '#000000';
	context.stroke();
	context.fillStyle= '#000000';
	context.fill();
	context.closePath();

	//EllissePupillaDestra
	context.beginPath();
	var centerX = 890.0;
	var centerY = 480.0;
	var radiusMax = 15.0;
	var radiusMin= 40.0;
	var rotation= 0.0*Math.PI/180;
	var startAngle=0.0*Math.PI/180;
	var endAngle=360.0*Math.PI/180;
	context.ellipse(centerX, centerY, radiusMax, radiusMin, rotation, startAngle, endAngle);
	context.lineWidth = 0.0;
	context.strokeStyle= '#000000';
	context.stroke();
	context.fillStyle= '#000000';
	context.fill();
	context.closePath();

	//CurvaSottoOcchi
	context.beginPath();
	context.lineWidth = 6.0;
	context.strokeStyle = '#000000';
	context.moveTo( 740.0, 552.0);
	context.quadraticCurveTo( 850.0, 470.0, 960.0, 552.0);
	context.stroke();
	context.fillStyle= '#FFFFFF';
	context.fill();
	context.closePath();

	//CirconferenzaCopriCurvaOcchiSinistra
	context.beginPath();
	var centerX = 740.0;
	var centerY = 552.0;
	var radius = 3.0;
	var startAngle = 0.0* Math.PI/180;
	var endAngle = 360.0* Math.PI/180;
	context.arc (centerX, centerY, radius, startAngle, endAngle);
	context.lineWidth = 0.0;
	context.strokeStyle= '#FFFFFF';
	context.stroke();
	context.fillStyle= '#FFFFFF';
	context.fill();
	context.closePath();

	//CirconferenzaCopriCurvaOcchiDestra
	context.beginPath();
	var centerX = 960.0;
	var centerY = 552.0;
	var radius = 3.0;
	var startAngle = 0.0* Math.PI/180;
	var endAngle = 360.0* Math.PI/180;
	context.arc (centerX, centerY, radius, startAngle, endAngle);
	context.lineWidth = 0.0;
	context.strokeStyle= '#FFFFFF';
	context.stroke();
	context.fillStyle= '#FFFFFF';
	context.fill();
	context.closePath();

	//CurvaMento
	context.beginPath();
	context.lineWidth = 2.0;
	context.strokeStyle = '#000000';
	context.moveTo( 700.0, 750.0);
	context.quadraticCurveTo( 850.0, 980.0, 1000.0, 750.0);
	context.stroke();
	context.fillStyle= '#FFFFFF';
	context.fill();
	context.closePath();

	//CurvaLabbroInferiore
	context.beginPath();
	context.lineWidth = 4.0;
	context.strokeStyle = '#000000';
	context.moveTo( 700.0, 696.0);
	context.quadraticCurveTo( 850.0, 990.0, 1000.0, 696.0);
	context.stroke();
	context.fillStyle= '#000000';
	context.fill();
	context.closePath();

	//CurvaLabbroSuperiore
	context.beginPath();
	context.lineWidth = 4.0;
	context.strokeStyle = '#000000';
	context.moveTo( 620.0, 635.0);
	context.quadraticCurveTo( 850.0, 850.0, 1080.0, 635.0);
	context.stroke();
	context.fillStyle= '#FFFFFF';
	context.fill();
	context.closePath();

	//CurvaLabbroSinistro
	context.beginPath();
	context.lineWidth = 3.0;
	context.strokeStyle = '#000000';
	context.moveTo( 598.0, 658.0);
	context.quadraticCurveTo( 625.0, 625.0, 653.0, 623.0);
	context.stroke();
	context.closePath();

	//CurvaLabbroDestro
	context.beginPath();
	context.lineWidth = 3.0;
	context.strokeStyle = '#000000';
	context.moveTo( 1102.0, 658.0);
	context.quadraticCurveTo( 1075.0, 625.0, 1047.0, 623.0);
	context.stroke();
	context.closePath();

	//EllisseNaso
	context.beginPath();
	var centerX = 850.0;
	var centerY = 600.0;
	var radiusMax = 65.0;
	var radiusMin= 50.0;
	var rotation= 0.0*Math.PI/180;
	var startAngle=0.0*Math.PI/180;
	var endAngle=360.0*Math.PI/180;
	context.ellipse(centerX, centerY, radiusMax, radiusMin, rotation, startAngle, endAngle);
	context.lineWidth = 0.0;
	context.strokeStyle= '#000000';
	context.stroke();
	context.fillStyle= '#000000';
	context.fill();
	context.closePath();

	//CerchioCopriMentoSinistro
	context.beginPath();
	var centerX = 700.0;
	var centerY = 773.0;
	var radius = 22.0;
	var startAngle = 0.0* Math.PI/180;
	var endAngle = 360.0* Math.PI/180;
	context.arc (centerX, centerY, radius, startAngle, endAngle);
	context.lineWidth = 0.0;
	context.strokeStyle= '#FFFFFF';
	context.stroke();
	context.fillStyle= '#FFFFFF';
	context.fill();
	context.closePath();

	//CerchioCopriMentoDestro
	context.beginPath();
	var centerX = 1000.0;
	var centerY = 773.0;
	var radius = 22.0;
	var startAngle = 0.0* Math.PI/180;
	var endAngle = 360.0* Math.PI/180;
	context.arc (centerX, centerY, radius, startAngle, endAngle);
	context.lineWidth = 0.0;
	context.strokeStyle= '#FFFFFF';
	context.stroke();
	context.fillStyle= '#FFFFFF';
	context.fill();
	context.closePath();

	//TerminaCirconferenzaViso
	context.beginPath();
	context.lineWidth = 0.0;
	context.strokeStyle = '#000000';
	context.moveTo( 1137.0, 642.0);
	context.quadraticCurveTo( 1137.0, 642.0, 1137.0, 641.0);
	context.stroke();
	context.closePath();

	//CerchioCopriTerminaCirconferenzaViso
	context.beginPath();
	var centerX = 1139.0;
	var centerY = 642.0;
	var radius = 1.0;
	var startAngle = 0.0* Math.PI/180;
	var endAngle = 360.0* Math.PI/180;
	context.arc (centerX, centerY, radius, startAngle, endAngle);
	context.lineWidth = 0.0;
	context.strokeStyle= '#FFFFFF';
	context.stroke();
	context.fillStyle= '#FFFFFF';
	context.fill();
	context.closePath();

	//EllisseLinguaSinistra
	context.beginPath();
	var centerX = 829.0;
	var centerY = 816.0;
	var radiusMax = 46.0;
	var radiusMin= 25.0;
	var rotation= 18.0*Math.PI/180;
	var startAngle=0.0*Math.PI/180;
	var endAngle=360.0*Math.PI/180;
	context.ellipse(centerX, centerY, radiusMax, radiusMin, rotation, startAngle, endAngle);
	context.lineWidth = 0.0;
	context.strokeStyle= '#FF0000';
	context.stroke();
	context.fillStyle= '#FF0000';
	context.fill();
	context.closePath();

	//EllisseLinguaDestra
	context.beginPath();
	var centerX = 867.0;
	var centerY = 817.0;
	var radiusMax = 46.0;
	var radiusMin= 25.0;
	var rotation= -18.0*Math.PI/180;
	var startAngle=0.0*Math.PI/180;
	var endAngle=360.0*Math.PI/180;
	context.ellipse(centerX, centerY, radiusMax, radiusMin, rotation, startAngle, endAngle);
	context.lineWidth = 0.0;
	context.strokeStyle= '#FF0000';
	context.stroke();
	context.fillStyle= '#FF0000';
	context.fill();
	context.closePath();

	//LineaLingua
	context.beginPath();
	context.lineWidth = 2.0;
	context.strokeStyle = '#000000';
	context.moveTo( 842.0, 792.0);
	context.quadraticCurveTo( 870.0, 803.0, 870.0, 815.0);
	context.stroke();
	context.closePath();

}
</script>
</head>
<body>
<canvas id='Topolino' width='1700.0' height='900.0'></canvas>
</body>
</html>
\end{verbatim}

\newpage

\subsection{Versione G8}
\begin{verbatim}
TITLE Topolino DRAWSPACE WIDTH 1700 DRAWSPACE HEIGTH 900
CIRC: NAME CirconferenzaViso XCENTER 850 YCENTER 550 RADIUS 300 COLORBODY #000000
ELLIPS: NAME EllisseOrecchioSinistro XCENTER 600 YCENTER 210 SEMIN 130 SEMAX 170 ROTATION -45
COLORBODY #000000
ELLIPS: NAME EllisseOrecchioDestro XCENTER 1100 YCENTER 210 SEMIN 130 SEMAX 170 ROTATION +45
COLORBODY #000000
ELLIPS: NAME EllisseParteAltaSinistraViso XCENTER 750 YCENTER 450 SEMIN 170 SEMAX 120 COLOR
#FFFFFF COLORBODY #FFFFFF
ELLIPS: NAME EllisseParteAltaDestraViso XCENTER 950 YCENTER 450 SEMIN 170 SEMAX 120 COLOR #FFFFFF
COLORBODY #FFFFFF
CIRC: NAME CerchioInferioreCoperturaNero XCENTER 850 YCENTER 550 RADIUS 299 STARTANGLE 16.2
ENDANGLE 163.8 COLOR #FFFFFF COLORBODY #FFFFFF
CURV: NAME CurvaGuanciaSinistra XSTART 566 YSTART 642 XMIDDLE 580 YMIDDLE 560 XEND 672 YEND 580
COLOR #FFFFFF COLORBODY #FFFFFF
CURV: NAME CurvaGuanciaDestra XSTART 1028 YSTART 580 XMIDDLE 1103 YMIDDLE 560 XEND 1136 YEND 642
COLOR #FFFFFF COLORBODY #FFFFFF
TRIANGLE: NAME TriangoloCoperturaNeroCentrale XA 566 YA 641 XB 1134 YB 641 XC 850 YC 476 COLOR
#FFFFFF COLORBODY #FFFFFF
ELLIPS: NAME EllisseOcchioSinistro XCENTER 788 YCENTER 450 SEMIN 100 SEMAX 40 WIDTH 3
ELLIPS: NAME EllisseOcchioDestro XCENTER 912 YCENTER 450 SEMIN 100 SEMAX 40 WIDTH 3
ELLIPS: NAME EllissePupillaSinistra XCENTER 810 YCENTER 480 SEMIN 40 SEMAX 15 COLORBODY #000000
ELLIPS: NAME EllissePupillaDestra XCENTER 890 YCENTER 480 SEMIN 40 SEMAX 15 COLORBODY #000000
CURV: NAME CurvaSottoOcchi XSTART 740 YSTART 552 XMIDDLE 850 YMIDDLE 470 XEND 960 YEND 552 WIDTH
6 COLORBODY #FFFFFF
CIRC: NAME CirconferenzaCopriCurvaOcchiSinistra XCENTER 740 YCENTER 552 RADIUS 3 COLOR #FFFFFF
COLORBODY #FFFFFF
CIRC: NAME CirconferenzaCopriCurvaOcchiDestra XCENTER 960 YCENTER 552 RADIUS 3 COLOR #FFFFFF
COLORBODY #FFFFFF
CURV: NAME CurvaMento XSTART 700 YSTART 750 XMIDDLE 850 YMIDDLE 980 XEND 1000 YEND 750 WIDTH 2
COLORBODY #FFFFFF
CURV: NAME CurvaLabbroInferiore XSTART 700 YSTART 696 XMIDDLE 850 YMIDDLE 990 XEND 1000 YEND 696
WIDTH 4 COLORBODY #000000
CURV: NAME CurvaLabbroSuperiore XSTART 620 YSTART 635 XMIDDLE 850 YMIDDLE 850 XEND 1080 YEND 635
WIDTH 4 COLORBODY #FFFFFF
CURV: NAME CurvaLabbroSinistro XSTART 598 YSTART 658 XMIDDLE 625 YMIDDLE 625 XEND 653 YEND 623
WIDTH 3
CURV: NAME CurvaLabbroDestro XSTART 1102 YSTART 658 XMIDDLE 1075 YMIDDLE 625 XEND 1047 YEND 623
WIDTH 3
ELLIPS: NAME EllisseNaso XCENTER 850 YCENTER 600 SEMIN 50 SEMAX 65 COLORBODY #000000
CIRC: NAME CerchioCopriMentoSinistro XCENTER 700 YCENTER 773 RADIUS 22 COLOR #FFFFFF COLORBODY
#FFFFFF
CIRC: NAME CerchioCopriMentoDestro XCENTER 1000 YCENTER 773 RADIUS 22 COLOR #FFFFFF COLORBODY
#FFFFFF
CURV: NAME TerminaCirconferenzaViso XSTART 1137 YSTART 642 XMIDDLE 1137 YMIDDLE 642 XEND 1137
YEND 641
CIRC: NAME CerchioCopriTerminaCirconferenzaViso XCENTER 1139 YCENTER 642 RADIUS 1 COLOR #FFFFFF
COLORBODY #FFFFFF
ELLIPS: NAME EllisseLinguaSinistra XCENTER 829 YCENTER 816 SEMIN 25 SEMAX 46 ROTATION +18 COLOR
#FF0000 COLORBODY #FF0000
ELLIPS: NAME EllisseLinguaDestra XCENTER 867 YCENTER 817 SEMIN 25 SEMAX 46 ROTATION -18 COLOR
#FF0000 COLORBODY #FF0000
CURV: NAME LineaLingua XSTART 842 YSTART 792 XMIDDLE 870 YMIDDLE 803 XEND 870 YEND 815 WIDTH 2
END
\end{verbatim}

\newpage

\subsection{Risultato grafico}
\includegraphics[scale=0.76]{Topolino.JPG} 

\end{document}